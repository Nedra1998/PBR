\documentclass[../pbr.tex]{subfile}

\begin{document}
\section{Outline}%
\label{sec:outline}

In practice the renderer is organized into several steps, these steps are
briefly outlined below, and the structure diagram is in
Figure.\ref{fig:p2_outline}.

\begin{figure}[htpb]
  \centering
  \digraph[scale=0.75]{outline}{
    rankdir=LR;
    splines=ortho;
    node [shape=box];
    subgraph cluster_library {
      subgraph cluster_shader {
        Sample;
        Evaluate;
        label="Shader";
        graph[style=dotted];
      }
      subgraph cluster_object {
        Distance;
        Normal;
        label="Object";
        graph[style=dotted];
      }
      subgraph cluster_camera {
        View;
        label="Camera";
        graph[style=dotted];
      }
      Renderer;
      WriteImage;
      RenderTile;
      RayMarch;
      label="Library";
      graph[style=dashed];
    }
    subgraph cluster_executable {
      subgraph cluster_user {
        Arguments;
        Scene;
        label="User";
        graph[style=dotted];
      }
      label="Executable";
      graph[style=dashed];
    }
    Arguments->Renderer;
    Scene->Renderer;
    Arguments->View;
    Scene->View;
    View->Renderer;
    Renderer->RenderTile;
    Renderer->WriteImage;
    RenderTile->RayMarch;
    RenderTile->Normal;
    RenderTile->Sample;
    RenderTile->Evaluate;
    RenderTile->Renderer;
    RayMarch->Distance;
    RayMarch->RenderTile;

  }
  \caption{Renderer Process Outline}%
  \label{fig:p2_outline}
\end{figure}

\begin{enumerate}
  \item \textit{User} These are the parts of the process that are defined and
    controlled by the user. These stages in the process only interact with the
    camera's view, and the renderer.
    \begin{enumerate}
      \item \textit{Arguments} These are the command line arguments specified by the
        user at runtime.
      \item \textit{Scene} This is the scene definition script defined by the user.
    \end{enumerate}
  \item \textit{Camera} These processes control and alter the camera in the
    scene.
    \begin{enumerate}
      \item \textit{View} This controls the cameras view, including its
        position, orientation, and projection.
    \end{enumerate}
  \item \textit{Renderer} This is the core process of the renderer. It handles
    constructing the scene, spawning the subprocesses and constructing the
    output files.
  \item \textit{RenderTile} This process is the core of the renderer, it
    handles detecting the intersections, processes the shading, and spawning
    recursive rays to march.
  \item \textit{RayMarch} This process detects the nearest intersection point
    of a ray in the scene, it is integral for the ray marching to function.
  \item \textit{Object} These stages are defined for every object, and define
    what the object is, these are the core of the ray marching process.
    \begin{enumerate}
      \item \textit{Distance} This is the core part of \textit{Ray Marching},
        this function defines the distance from any point to the given object.
      \item \textit{Normal} This is simply used to numerically approximate the
        normal of the surface of an object at a given point.
    \end{enumerate}
  \item \textit{Shader} These stages define the look of each object, and are at
    the core of the path tracing process.
    \begin{enumerate}
      \item \textit{Sample} This stage is used to greatly improve the
        efficiency of path tracing, is selects the reflected/refracted ray
        based upon importance sampling from the material definition.
      \item \textit{Evaluate} This process evaluates the color of light
        traveling along the view ray, due to the sampled incident ray.
    \end{enumerate}
\end{enumerate}

\end{document}
