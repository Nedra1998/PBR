\documentclass[../pbr.tex]{subfile}

\begin{document}
\section{User Control}%
\label{sec:user_control}

The user controls were a critical component of the Specula library. One of the
key goals of the library was to be easily used, and remove as many complications
as possible from the process of creating a scene. The balance from providing
the user with all the power provided by a path marcher, while at the same time
creating a simple system was remedied, by allowing the user direct access to
the core library, and developing an interface system that handles most of the
boilerplate code.

\subsection{Command Line Arguments}%
\label{sub:command_line_arguments}

This is the first method that the user will interact with the renderer, is
through the command line arguments. To develop this, Specula makes use of an
open source library CLI11\footnote{\url{https://github.com/CLIUtils/CLI11}}.
This library provides a generic command line argument parser, and several
validators that the code makes use of. The validators are used to restrict the
types of values that can be entered, thus restricting the arguments to sensible
values.

The CLI11 library provides most of the validators, such as an
\mintinline{cpp}{CLI::ExistingFile} and \mintinline{cpp}{CLI::PositiveNumber}
validators. However, for some arguments a more generalizable validator was
required, so using the inheritance of the base validator
(\mintinline{cpp}{CLI::Validator}), the implementation of a
\mintinline{cpp}{RegexValidator} was made. The regex validator takes a regular
expression at construction, and will test the users input against that pattern,
and if the input fails the match the pattern, then an error is thrown.

Parsing the command line arguments is the first step that happens in the
entirety of the code, this way if an argument is invalid then the program
exists before processing anything. Once these arguments have been parsed, then
they can be used throughout the rest of the process, by any component in the
executable. It is important to note that is is entirely separated from all of
the library code. This means that it is possible to utilize the library without
including this system for parsing command line arguments.

The specifics of the command line arguments are further described in Section
\ref{sec:command_line_arguments}.

\subsection{Scene Description}%
\label{sub:scene_description}

The primary form of interface that the user will have with the library, is with
the scene definition script. The scene definition script is specified at run
time, and is a \texttt{Lua} script that the user uses to define the scene's
geometry and properties.

Having the scene defined in a \texttt{Lua} script, means that all of
\texttt{Lua}'s standard libraries are available to the user, and the scene
description can be generated at runtime through the script. The choice to use
\texttt{Lua} was primarily based upon its ease of integration with
\texttt{C/C++}, and its performance. \texttt{Lua} usually executes faster than
most scripting languages, and there are further improvements that can be made
to improve the performance even more.

To integrate the existing libraries API with the Lua scripting language, we use
two open source tools. Firstly LuaJIT\footnote{\url{http://luajit.org/}} is a
\textit{Just In Time} compiler for lua. This means that at execution time, the
lua code is compiled into byte code for improved efficiency and performance.
Secondly we use Sol2\footnote{\url{https://github.com/ThePhD/sol2}}. Sol2 is a
library that abstracts the interface with the lua interpreter provided by
LuaJIT. With these too tools, it is possible to provide almost all of the
available function from the \texttt{C++} API to the lua interface. This means
that all of the object primitives that have been defined for use, are also
available as classes in the Lua interpreter.

The key difference between the \texttt{C++} interface, and the \texttt{Lua}
interface, is that when the Lua version of the \mintinline{lua}{render()} is
called, then the process recognizes that the intention is to render a sequence
of images, and handles that accordingly. If \mintinline{lua}{render()} is never
called in the scene description script, then the process assumes that a single
image was intended to be rendered, and renders the state of the scene after
completing of the scene description execution. In the \texttt{C++} interface it
is required that the \mintinline{cpp}{render()} function is explicitly called.

Similarly to the command line arguments, all of the scene description and the
\texttt{Lua} scripting is contained within the executable process, and will not
directly interact with the library. This means that only the executable
requires the LuaJIT and Sol2 libraries, which are of significant size, thus
reducing the size of the core Specula library.

The specifics of the API that is available in the scene description script is
explained in Section \ref{sec:scene_file}.

\end{document}
