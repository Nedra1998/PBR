\documentclass[../pbr.tex]{subfile}

\begin{document}
\section{Command Line Arguments}%
\label{sec:command_line_arguments}

The command line arguments can be used to define global scene parameters. These
options primarily specify the output of the program, and can also control some
global parameters for the renderer.

\subsection{Options}%
\label{sub:options}

These options do not effect the renderer in any way, and are used for debugging
information, and to get a help message.

\begin{description}
  \item[\texttt{-h,--help}] Displays built in help message.
  \item[\texttt{-v}] Changes the verbosity of the output. Include more of this
    flag for a more verbose output. The range is from $0-6$, where $6$ will
    print all levels of messages.
\end{description}

\subsection{Output}%
\label{sub:output}

These options directly effect what image files are generated, and the type and
size of these output images.

\begin{description}
    \item[\texttt{--aspect RATIO}] Defines the aspect ratio of the image to
      generate. This must be of the form \texttt{w:h}, or \texttt{wxh}, any
      other formats will cause an error. Here \texttt{w} and \texttt{h} can be
      either floating point number or integer numbers, but must be positive.
    \item[\texttt{-a,--albedo}] Enabling this flag will generate additional
      files with the file path based upon the output file path
      \texttt{"file-albedo.extension"}, which contain the unshaded colors of
      the scene. This is primarily useful for denoising the image after
      generation.
    \item[\texttt{-d,--depth}] Enabling this flag will generate additional
      files with the file path based upon the output file path
      \texttt{"file-depth.extension"}, which contain the depth map of the
      scene. This is primarily useful for denoising the image after generation.
    \item[\texttt{-n,--normal}] Enabling this flag will generate additional
      files with the file path based upon the output file path
      \texttt{"file-normal.extension"}, which contain the normal directions of
      the scene. This is primarily useful for denoising the image after
      generation.
    \item[\texttt{-o,--output FILE}] Output file and directory. If a single
      frame is being rendered, then this will be the path to the file. If
      multiple frames are being rendered, then this argument will be the
      directory to generate the numbered files in and the extension for each
      file. Valid extensions to generate are \texttt{png}, \texttt{jpeg}, and
      \texttt{bmp}, these are the only extensions that have been defined for
      the image generator, all other extensions will cause an error.
    \item[\texttt{-r,--res,--resolution WIDTH}] This defines the output
      resolution width of the image. Combined with the usage of
      \texttt{--aspect}, this allows the construction of any size image file.
\end{description}

\subsection{Renderer}%
\label{sub:renderer}

These options effect the renderer, and can greatly improve or hurt the
performance or can change how the output looks.

\begin{description}
  \item[\texttt{-b,--bounces NUM}] This variable controls the minimum number of
    bounces that the renderer will simulate. More bounces will produce a more
    realistic result, but will also significantly slow down the simulation.
    Note that this only defines the minimum number of bounces, most rays will
    bounce $5-10$ times more than this value. This defaults to $10$.
  \item[\texttt{-f,--fov FOV}] This defines the horizontal field of view for
    the generated image. The vertical field of view is calculated based upon
    the aspect ratio. For this argument \texttt{FOV} must be a floating point
    number, representing the radians for the field of view. This argument
    defaults to $45^\circ$.
  \item[\texttt{--no-denoise}] This flag will disable the built in denoiser.
    Note that has not yet been implemented, no denoiser will be used.
  \item[\texttt{-s,--spp SAMPLES}] This will determine the number of samples to
    render for each pixel of the image. A higher value produces significantly
    better resulting images, but will greatly increase the render time. This
    value defaults to $16$.
  \item[\texttt{-t,--tile SIZE}] This argument defines the size of the tile,
    and consequently the maximum number threads to use. Each rendered image is
    divided into multiple tiles, and this argument defines the size of the
    tile. This defaults to $16$, meaning that each tile is $16\times 16$
    pixels.
\end{description}

\end{document}
