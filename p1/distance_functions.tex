\documentclass[../pbr.text]{subfile}

\begin{document}
\section{Distance Functions}%
\label{sec:distance_functions}

For the ray marching method, each object must have a function that given a
point returns the distance from that point to the object. In this section we
will provide distance functions for many primitives, and some formulas for
combining primitives together into more complex shapes. Even more complex
shapes can have a distance function such as more complex fractals.

\subsection{Primitives}%
\label{sub:primitives}

All primitives are centered at the origin, any scaling translation or rotation
must be done separately.

\subsubsection{Sphere}%
\label{ssub:sphere}
\begin{minted}{glsl}
  return length(p)-radius;
\end{minted}

\subsubsection{Box}%
\label{ssub:box}
\begin{minted}{glsl}
  vec3 q = abs(p) - b;
  return length(max(q,0.0)) + min(max(q.x,max(q.y,q.z)),0.0);
\end{minted}

\subsubsection{Round Box}%
\label{ssub:round_box}
\begin{minted}{glsl}
  vec3 q = abs(p) - b;
  return length(max(q,0.0)) + min(max(q.x,max(p.y,q.z)),0.0) - r;
\end{minted}

\subsubsection{Torus}%
\label{ssub:torus}
\begin{minted}{glsl}
  vec2 q = vec2(length(p.xz)-t.x,p.y);
  return length(q) - t.y;
\end{minted}

\subsubsection{Plane}%
\label{ssub:plane}
\begin{minted}{glsl}
  return dot(p,n.xyz)) + n.w;
\end{minted}

\subsubsection{Cylinder}%
\label{ssub:cylinder}
\begin{minted}{glsl}
  vec2 d = abs(vec2(length(p.xz),p.y)) - vec2(h,r);
  return min(max(d.x, d.y), 0.0) + length(max(d, 0.0));
\end{minted}

\subsubsection{Rounded Cylinder}%
\label{ssub:rounded_cylinder}
\begin{minted}{glsl}
  vec2 d = vec2(length(p.xz)-2.0*ra+rb, abs(p.y)-h);
  return min(max(d.x, d.y), 0.0) + length(max(d, 0.0)) - rb;
\end{minted}

\subsubsection{Cone}%
\label{ssub:cone}
\begin{minted}{glsl}
  vec2 q = vec2( length(p.xz), p.y );
  vec2 k1 = vec2(r2,h);
  vec2 k2 = vec2(r2-r1,2.0*h);
  vec2 ca = vec2(q.x-min(q.x,(q.y<0.0)?r1:r2), abs(q.y)-h);
  vec2 cb = q - k1 + k2*clamp( dot(k1-q,k2)/dot2(k2), 0.0, 1.0 );
  float s = (cb.x<0.0 && ca.y<0.0) ? -1.0 : 1.0;
  return s*sqrt( min(dot2(ca),dot2(cb)) );
\end{minted}

\subsubsection{Round Cone}%
\label{ssub:round_cone}
\begin{minted}{glsl}
  vec2 q = vec2( length(p.xz), p.y );

  float b = (r1-r2)/h;
  float a = sqrt(1.0-b*b);
  float k = dot(q,vec2(-b,a));

  if( k < 0.0 ) return length(q) - r1;
  if( k > a*h ) return length(q-vec2(0.0,h)) - r2;

  return dot(q, vec2(a,b) ) - r1;
\end{minted}

\subsubsection{Triangle}%
\label{ssub:triangle}
\begin{minted}{glsl}
  vec3 ba = b - a; vec3 pa = p - a;
  vec3 cb = c - b; vec3 pb = p - b;
  vec3 ac = a - c; vec3 pc = p - c;
  vec3 nor = cross( ba, ac );

  return sqrt(
  (sign(dot(cross(ba,nor),pa)) +
  sign(dot(cross(cb,nor),pb)) +
  sign(dot(cross(ac,nor),pc))<2.0)
  ?
  min( min(
  dot2(ba*clamp(dot(ba,pa)/dot2(ba),0.0,1.0)-pa),
  dot2(cb*clamp(dot(cb,pb)/dot2(cb),0.0,1.0)-pb) ),
  dot2(ac*clamp(dot(ac,pc)/dot2(ac),0.0,1.0)-pc) )
  :
  dot(nor,pa)*dot(nor,pa)/dot2(nor) );
\end{minted}

\subsubsection{Quad}%
\label{ssub:quad}
\begin{minted}{glsl}
  vec3 ba = b - a; vec3 pa = p - a;
  vec3 cb = c - b; vec3 pb = p - b;
  vec3 dc = d - c; vec3 pc = p - c;
  vec3 ad = a - d; vec3 pd = p - d;
  vec3 nor = cross( ba, ad );

  return sqrt(
  (sign(dot(cross(ba,nor),pa)) +
  sign(dot(cross(cb,nor),pb)) +
  sign(dot(cross(dc,nor),pc)) +
  sign(dot(cross(ad,nor),pd))<3.0)
  ?
  min( min( min(
  dot2(ba*clamp(dot(ba,pa)/dot2(ba),0.0,1.0)-pa),
  dot2(cb*clamp(dot(cb,pb)/dot2(cb),0.0,1.0)-pb) ),
  dot2(dc*clamp(dot(dc,pc)/dot2(dc),0.0,1.0)-pc) ),
  dot2(ad*clamp(dot(ad,pd)/dot2(ad),0.0,1.0)-pd) )
  :
  dot(nor,pa)*dot(nor,pa)/dot2(nor) );
\end{minted}

\subsection{Alterations}%
\label{sub:alterations}

It is possible to apply operation that will change the shape of the primitives,
while retaining the exact euclidean metric to them.

\subsubsection{Rounding}%
\label{ssub:rounding}
\begin{minted}{glsl}
  return primative(p) - radius;
\end{minted}

\subsubsection{Onion}%
\label{ssub:onion}
\begin{minted}{glsl}
  return abs(primative(p)) - thickness;
\end{minted}

\subsection{Combinations}%
\label{sub:combinations}

These function combine, carve, or intersect the different primitives. These
function can take the result of any two distance functions we will denote these
values as \texttt{d1} and \texttt{d2}.

\subsubsection{Union}%
\label{ssub:union}
\begin{minted}{glsl}
  return min(d1,d2);
\end{minted}

\subsubsection{Subtraction}%
\label{ssub:subtraction}
\begin{minted}{glsl}
  return max(-d1,d2);
\end{minted}

\subsubsection{Intersection}%
\label{ssub:intersection}
\begin{minted}{glsl}
  return max(d1,d2);
\end{minted}

\subsubsection{Smooth Union}%
\label{ssub:smooth_union}
\begin{minted}{glsl}
  float h = clamp( 0.5 + 0.5*(d2-d1)/k, 0.0, 1.0 );
  return mix( d2, d1, h ) - k*h*(1.0-h);
\end{minted}

\subsubsection{Smooth Subtraction}%
\label{ssub:smooth_subtraction}
\begin{minted}{glsl}
  float h = clamp( 0.5 - 0.5*(d2+d1)/k, 0.0, 1.0 );
  return mix( d2, -d1, h ) + k*h*(1.0-h);
\end{minted}

\subsubsection{Smooth Intersection}%
\label{ssub:smooth_intersection}
\begin{minted}{glsl}
  float h = clamp( 0.5 - 0.5*(d2-d1)/k, 0.0, 1.0 );
  return mix( d2, d1, h ) + k*h*(1.0-h);
\end{minted}

\subsection{Mandelbulb}%
\label{sub:mandelbulb}

Using the format of a distance function it is possible to generate more complex
objects, like this example is the distance function to render the mandelbulb.
\begin{minted}{glsl}
  vec3 z = p;
  float dr = 1.0;
  float r = 0.0;
  for (int i = 0; i < Iterations ; i++) {
    r = length(z);
    if (r>Bailout) break;

    // convert to polar coordinates
    float theta = acos(z.z/r);
    float phi = atan(z.y,z.x);
    dr =  pow( r, Power-1.0)*Power*dr + 1.0;

    // scale and rotate the point
    float zr = pow( r,Power);
    theta = theta*Power;
    phi = phi*Power;

    // convert back to cartesian coordinates
    z = zr*vec3(sin(theta)*cos(phi), sin(phi)*sin(theta), cos(theta));
    z+=p;
  }
  return 0.5*log(r)*r/dr;
\end{minted}

\end{document}
