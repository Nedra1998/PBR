\documentclass[../pbr.tex]{subfile}

\begin{document}
\section{Shading}%
\label{sec:shading}

Being able to calculate light ray intersection points with the objects in the
scene is only half the process. The other half is calculating the color of
light, and the shading of the objects. To explain this we will first introduce
the core concept of rendering called the Rendering Equations.

\subsection{Rendering Equation}%
\label{sub:rendering_equation}

The core of all rendering is the rendering equation. The explicit form of this
equation is shown in \ref{eq:render}.
\begin{align}\label{eq:render}
  L_o(\mathbf{x},\omega_o,\lambda,t)=L_e(\mathbf{x},\omega_o,\lambda,t)+\int_\Omega
  f_{bsdf}(\mathbf{x},\omega_i,\omega_o,\lambda,t)L_i(\mathbf{x},\omega_i,\lambda,t)(\omega_i\cdot\mathbf{n})d\omega_i
\end{align}
Breaking this equation into these components is necessary for understanding
it.
\begin{description}
  \item[$\mathbf{x}$] is the location in space. This is the point of
    intersection that we compute earlier in the process.
  \item[$\omega_o$] is the direction of outgoing \textit{light}. Note that this
    is the direction of the light, which as we explained earlier is the opposite
    direction from the ray.
  \item[$\omega_i$] is the negative direction of incoming light. Since this is
    the negative direction of the light, and the light is the negative
    direction of our rays, then we can use the direction of the
    reflected/refracted rays for this value.
  \item[$\lambda$] is the wavelength of light that we are considering. This is
    present as each wavelength of light will interact with a surface slightly
    differently. However in our simulation we will only consider the standard
    red, green, blue values of light. This restriction should be removed in a
    later version of the software.
  \item[$t$] is the time.
  \item[$\Omega$] is the unit sphere around the point $\mathbf{x}$, containing
    all possible values for $\omega_i$.
  \item[$\mathbf{n}$] is the surface normal at $\mathbf{x}$.
  \item[$L_o$] is the total \textit{spectral radiance} for the wavelength
    $\lambda$ in the outward direction $\omega_o$, from the position
    $\mathbf{x}$ at time $t$.
  \item[$L_e$] is the emitted spectral radiance of the point $\mathbf{x}$, this
    will only come into play if the object is marked as missive such as
    lights.
  \item[$f_{bsdf}$] Is the bidirectional scattering distribution function.
    Generally this function determines the probability that a ray of light from
    the direction $\omega_i$ will be reflected or refracted to the ray of
    $\omega_o$. If this value is zero, then there should be no contribution
    from the light along $\omega_i$ to the light along $\omega_o$.
  \item[$L_i$] is the spectral radiance coming inward toward $\mathbf{x}$ from
    direction $\omega_i$. To compute this value we would follow the ray in the
    direction $\omega_i$ until an object is intersected. Then we must compute
    $L_o$ for that point, with our $\omega_i$ as a new $\omega_o$.
\end{description}

\end{document}
