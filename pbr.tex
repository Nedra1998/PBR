\documentclass[10pt]{armath}
\usepackage[utf8]{inputenc}
\usepackage[english]{babel}
\usepackage{amsmath}
\usepackage{amsthm}
\usepackage{csquotes}
\usepackage{enumitem}
\usepackage{subcaption}
\usepackage{todonotes}
\usepackage{multicol}
\usepackage{tikz}
\usepackage{pgfplots}
\usepackage{subfiles}
\usepackage{longtable}
\usepackage{fontawesome}
\usepackage{verbatim}
\usepackage[pdf]{graphviz}
\usepackage[toc,page]{appendix}
\usepackage[hidelinks]{hyperref}

\usepackage{minted}

\title{Physically Based Rendering}
\author{Arden Rasmussen}
\date{\today}

\numberwithin{equation}{section}
\newcommand{\hdiv}[3]{
  \vspace{#1}%
  \noindent\rule{\textwidth}{#2}%
  \vspace{#3}%
}
\usepackage{relsize}

\tikzset{fontscale/.style = {font=\relsize{#1}}}

\begin{document}
\maketitle

\begin{abstract}
  Physically Based Rendering (PBR) is the method for rendering computer
  generated scenes, in a way that is intended to mimic that of reality. The
  primary goal of a PBR system, is to simulate every ray of light and its
  interaction with the scene.  The goal of the system that is described in this
  paper is to make a PBR system with extremely simple user interface, and one
  that uses the method of ray marching. The library and executable are open
  source and are available on
  Github\footnote{\url{https://github.com/LuxAter/Specula}}.
\end{abstract}

\newpage
\hdiv{10pt}{0.5pt}{1pt}
\part{Theory}%
\label{prt:theory}
\hdiv{1pt}{0.5pt}{10pt}

\subfile{p1/light1}
\subfile{p1/camera}
\subfile{p1/marching}

\newpage
\hdiv{10pt}{0.5pt}{1pt}
\part{Practice}%
\label{prt:practice}
\hdiv{1pt}{0.5pt}{10pt}

\subfile{p2/outline}
\subfile{p2/user}

\newpage
\hdiv{10pt}{0.5pt}{1pt}
\part{Analysis}%
\label{prt:Analysis}
\hdiv{1pt}{0.5pt}{10pt}

\newpage
\hdiv{10pt}{0.5pt}{1pt}
\part{Usage}%
\label{prt:Usage}
\hdiv{1pt}{0.5pt}{10pt}

\subfile{p4/args}
\subfile{p4/file}

\end{document}
